\documentclass{article}
\usepackage{graphicx}
\usepackage{amsfonts}
% \usepackage{fancyhdr}
\usepackage[a4paper, margin=0.5in]{geometry}
\usepackage{amsmath}
\usepackage{amssymb}
\usepackage{enumitem}


% \usepackage{amsmath}

\title{Week 5 Homework}
\author{Bryan SebaRaj \\[0.7em] Professor Hong Tang \\[0.7em]  EENG 203 - Circuits and System Design}
\date{February 18, 2025}
\begin{document}



\maketitle

\section*{Homework for February 11, 2025}

\subsection*{14.61}
% \includegraphics[scale=0.5]{/Users/bryansebaraj/Desktop/Screenshot 2025-02-16 at 2.20.40 PM.png}

\begin{enumerate}[label=(\alph*)]
    \item Note that in this active filter, $$V_o=V_- \text{ and } V_+=
        V_i \frac{\frac{1}{j\omega C}}{R+\frac{1}{j \omega C}}$$
        Recall that $V_+=V_-$. As such, 
        $$V_i \frac{\frac{1}{j\omega C}}{R+\frac{1}{j \omega C}}=V_o$$
        Thus, $$H(\omega)=\frac{V_o}{V_i}=\frac{j\omega C}{j\omega C}\cdot \frac{\frac{1}{j\omega C}}{R+\frac{1}{j \omega C}}=\frac{1}{j\omega RC + 1}$$
    \item Note that in this active filter, $$V_o=V_- \text{ and } V_+=V_i\frac{R}{R + \frac{1}{j\omega C}}$$
        Recall that $V_+=V_-$. As such,
        $$V_i\frac{R}{R + \frac{1}{j\omega C}}=V_o$$
        Thus, $$H(\omega)=\frac{V_o}{V_i}=\frac{j\omega C}{j \omega C} \cdot \frac{R}{R + \frac{1}{j\omega C}}=\frac{j\omega RC}{j\omega RC + 1}$$
\end{enumerate}


\subsection*{14.66}
\includegraphics[scale=0.8]{/Users/bryansebaraj/Desktop/Screenshot 2025-02-16 at 2.21.03 PM.png}

\begin{enumerate}[label=(\alph*)]
    \item Let the output of the op-amp be denoted as $v_a$ and the node between $R_3$ and $R_4$ be denoted with a voltage $v_b$, the non-inverting input as $v_+$ and the inverting input as $v_-$. 
        Trivially, $v_+=v_b$. From inspection of the resistor divider between $v_a$ and ground, 
        $$v_b=\frac{R_4}{R_3 + R_4}v_a=v_+$$
        Recall that $v_-=v_+$. Thus,
        $$v_-=\frac{R_4}{R_3 + R_4}v_a$$
        Using KCL at the inverting input node, 
        $$\frac{v_s - v_-}{R_1} + \frac{v_a - v_-}{R_2} + C(s(v_a - v_-)) = 0$$
        Substituting $v_-$,
        $$\frac{v_s - \frac{R_4}{R_3 + R_4}v_a}{R_1} + \frac{v_a - \frac{R_4}{R_3 + R_4}v_a}{R_2} + C(s(v_a - \frac{R_4}{R_3 + R_4}v_a)) = 0$$
    \item Note that in order to operate as a high pass filter, $H(s)$ must take the form with some constant $K$,
        $$H(s)=K\cdot \frac{sR_fC}{sR_iC + 1}$$
        Rewriting $H(s)$, $$H(s)=\frac{R_4}{R_3 + R_4} \times \frac{R_2C}{R_2C} \times \frac{s + (1/R_1C)[R_1/R_2 - R_3/R_4]}{s + 1/R_2C}=\frac{R_4}{R_3 + R_4} \times \frac{sR_2C + (R_2/R_1)[R_1/R_2 - R_3/R_4]}{sR_2C + 1}$$
        When $\frac{R_1}{R_2}=\frac{R_4}{R_4}$, $H(s)$ takes the form,
        $$H(s)=\frac{R_4}{R_3 + R_4} \times \frac{sR_2C + 0}{sR_2C + 1}=\frac{R_4}{R_3 + R_4} \times \frac{sR_2C}{sR_2C + 1}$$
        Thus, in order for this circuit to act as a high pass filter, $\frac{R_1}{R_2}=\frac{R_3}{R_4}$.
        This can be confirmed as $H(0)=0$ and $H(\infty)=1$.
    \item 
        % To operate as a lowpass filter, $H(s)$ must take the form,
        % $$H(s)=K \frac{1}{sR_2C + 1}$$
        Using the previous intermerdiate form, $$H(s)=\frac{R_4}{R_3 + R_4} \times \frac{sR_2C + (R_2/R_1)[R_1/R_2 - R_3/R_4]}{sR_2C + 1}$$
        At $s=0$, $$H(0)=\frac{R_4}{R_3 + R_4} \times \frac{0 + (R_2/R_1)[R_1/R_2 - R_3/R_4]}{0 + 1}=\frac{R_4}{R_3 + R_4} (R_2/R_1)(R_1/R_2 - R_3/R_4)$$
        Therefore in order to yield a nonzero DC gain at at $s=0$, $$\frac{R_1}{R_2} \neq \frac{R_3}{R_4}$$
        In order for $H(\infty)=0$, see that $R_3\rightarrow \infty$, which would yield,
        $$H(\infty)=0$$
        Thus, the circuit will act as a low pass filter.

\end{enumerate}


\section*{Homework for February 13, 2025}

\subsection*{14.95}
% \includegraphics[scale=0.9]{/Users/bryansebaraj/Desktop/Screenshot 2025-02-16 at 2.21.15 PM.png}

\begin{enumerate}[label=(\alph*)]
    \item Recall that a circuit with a variable capacitor and antenna coil uses the formula, 
        $$f=\frac{1}{2\pi \sqrt{LC}}$$
        At $C=40 \text{ pF}$, $$f=\frac{1}{2\pi \sqrt{240 \mu H \cdot 40 pF}}=1.6244 \text{ MHz}$$
        At $C=360 \text{ pF}$, $$f=\frac{1}{2\pi \sqrt{240 \mu H \cdot 360 pF}}=0.5415 \text{ MHz}$$
        Thus, the circuit can tune for any frequency between $0.5415 \text{ MHz}$ and $1.6244 \text{ MHz}$.
    \item In order to calculate $Q$, recall that the formula for $Q$ is,
        $$Q=\frac{2\pi f L}{R}$$
        At $f_1=0.5415 \text{ MHz}$, $$Q=\frac{2\pi \cdot 0.5415 \text{ MHz} \cdot 240 \mu H}{12 \Omega}=68.047$$
        At $f_2=1.6244 \text{ MHz}$, $$Q=\frac{2\pi \cdot 1.6244 \text{ MHz} \cdot 240 \mu H}{12 \Omega}=204.128$$

\end{enumerate}


\subsection*{14.96}
\includegraphics[scale=0.9]{/Users/bryansebaraj/Desktop/Screenshot 2025-02-16 at 2.21.21 PM.png}

First let us denote $Z_1$ as the node directly below $C_1$ and $Z_2$ as the node directly below $C_2$ Using nodal analysis, see that $$Z_2=\frac{R_L\cdot \frac{1}{sC_2}}{R_L + \frac{1}{sC_2}}=\frac{R_L}{R_LsL + 1}$$ and
$$Z_1=\frac{1}{sL} || (sL + Z_2)=\frac{1}{sC} || \left( \frac{sL + s^2R_LC_2L + R_L}{sR_LC}\right)=\frac{sL+s^2R_LC_2L+R_L}{1 + sR_LC_2 + s^2LC_1 + s^3R_LLC_1C_2 + sC_1R_L}$$
Analyzing $V_1$ and $V_0$, see that $$V_1 = \frac{Z_1}{Z_1 + R_i}V_i$$
$$V_0 = \frac{Z_2}{Z_2 + sL}V_1=\frac{Z_2}{Z_2 + sL} \frac{Z_1}{Z_1 + R_i}V_i$$
$$=\frac{sL + s^2R_LLC_2+R_L}{sL + s^2R_LLC_2 + R_i(1 + sR_LC_2 + s^2LC_1 + s^3R_LLC_1C_2 + sC_1R_L)} \cdot \frac{R_L}{R_L + sL(1 + sR_LC_2)}V_i$$
Thus, $$H(\omega)=\frac{V_o}{V_i}=\frac{R_LsL + s^2R_L^2LC_2+R_L^2}{(sL + s^2R_LLC_2 + R_i + R_isR_LC_2 + R_is^2LC_1 + R_is^3R_LLC_1C_2 + R_isC_1R_L)(R_L + sL + s^2LR_LC_2)}$$

\end{document}

