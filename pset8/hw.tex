\documentclass{article}
\usepackage{graphicx}
\usepackage{amsfonts}
% \usepackage{fancyhdr}
\usepackage[a4paper, margin=0.5in]{geometry}
\usepackage{amsmath}
\usepackage{amssymb}
\usepackage{enumitem}


% \usepackage{amsmath}

\title{Week 8 Homework}
\author{Bryan SebaRaj \\[0.7em] Professor Hong Tang \\[0.7em]  EENG 203 - Circuits and System Design}
\date{April 3, 2025}
\begin{document}


\maketitle

\section*{Homework for March 27, 2025}

\subsection*{16.35}
% \includegraphics[scale=0.9]{/Users/bryansebaraj/Desktop/Screenshot 2025-03-29 at 4.25.29 PM.png}


For $t<0$, the capacitor and inducotr are in a steady state at 0 as the circuit is unexicted. Transforming the given circuit yeilds

\begin{center}
    \includegraphics[scale=0.15]{/Users/bryansebaraj/Downloads/IMG_169210321C82-1.jpeg}
\end{center}
%
% \[
%     \frac{V_o - \frac{10}{s+1}}{s} + \frac{2s(V_o - 0)}{1} + \frac{V_o - 0}{4} - \frac{3}{s} = 0
% \]
%
% \[
% \left( \frac{1}{s} + 2s + \frac14 \right)V_o = \frac{10}{s(s+1)} + \frac{3}{s}
% \]
%
% Solving for $V_o$ yields
%
% \[
% 2 \left( \frac{s^2 + 0.125s + 0.5}{s} \right)V_o = \frac{3s + 13}{s(s+1)}
% \]
% \[
% V_o = \frac{1.5s + 6.5}{(s^2 + 0.125s + 0.5)(s+1)}
% \]
%
% Find the roots of $s^2 + 0.125s + 0.5$,
%
% \[
% s_{1,2} = \frac{-0.125 \pm \sqrt{0.015625 - 2}}{2} = \frac{-0.125 \pm \sqrt{-1.984375}}{2} = -0.0625 \pm j0.70435
% \]
%
% Substituting those roots back into $V_o$ yields
%
% \[
% V_o = \frac{1.5s + 6.5}{(s + 1)(s + 0.0625 + j0.70435)(s + 0.0625 - j0.70435)}
% \]
%
% \[
% = \frac{A}{s + 1} + \frac{B}{s + 0.0625 + j0.70435} + \frac{C}{s + 0.0625 - j0.70435}
% \]
%
% where
% \[
% A = \frac{1.5s + 6.5}{(s + 0.0625 + j0.70435)(s + 0.0625 - j0.70435)}\Bigg|_{s = -1} = \frac{5}{1.375} = 3.636
% \]
%
% \[
% B = \frac{(1.5(-0.0625 + j0.70435) + 6.5)}{(-0.0625 + j0.70435 + 1)(-1.4087)} = \frac{6.40625 - j1.08625}{(0.9375 + j0.70435)(-1.4087)}
% \]
%
% \[
% = \frac{(6.49272 \angle -9.36497^\circ)}{(1.17261 \angle 36.9178^\circ)(1.4087 \angle -90^\circ)} = 3.9306 \angle 117.553^\circ
% \]
%
% \[
% C = \frac{(1.5(-0.0625 - j0.70435) + 6.5)}{(-0.0625 - j0.70435 + 1)(-1.4087)} = \frac{6.40625 + j1.056525}{(0.9375 - j0.70435)(-1.4087)}
% \]
%
% \[
% = \frac{6.49272 \angle 9.36497^\circ}{(1.17261 \angle -36.9178^\circ)(1.4087 \angle 90^\circ)} = 3.9306 \angle -117.553^\circ
% \]
%
% Thus,
%
% \[
% v_o(t) = (3.636 e^{-t} + 3.9306 e^{-0.0625t}e^{j117.559^\circ} e^{-j0.7044t} + 3.9306 e^{-0.0625t}e^{-j117.559^\circ} e^{j0.7044t})u(t) \text{ V}
% \]
%
% or
%
% \[
% [3.636 e^{-t} + 7.862 e^{-0.0625t} \cos(0.7044t - 117.55^\circ)]u(t) \text{ volts}
% \]
The s-domain equation for KCL at the node $v_o$ is:
$$\frac{v_o(s) - \frac{10}{s+1}}{s} + \frac{v_o(s)}{4} + 2s\cdot v_o(s) = \frac{3}{s}$$

$$\frac{v_o(s)}{s} - \frac{10}{s(s+1)} + \frac{v_o(s)}{4} + 2s\cdot v_o(s) = \frac{3}{s}$$

Multiply all terms by $s$,
$$v_o(s) - \frac{10}{s+1} + \frac{s\cdot v_o(s)}{4} + 2s^2\cdot v_o(s) = 3$$

$$v_o(s) + \frac{s\cdot v_o(s)}{4} + 2s^2\cdot v_o(s) = 3 + \frac{10}{s+1}$$

$$v_o(s)\left(1 + \frac{s}{4} + 2s^2\right) = 3 + \frac{10}{s+1}$$

$$1 + \frac{s}{4} + 2s^2 = \frac{4 + s + 8s^2}{4} = \frac{8s^2 + s + 4}{4}$$

Therefore,
$$v_o(s) = \frac{3 + \frac{10}{s+1}}{\frac{8s^2 + s + 4}{4}}$$
$$v_o(s) = \frac{4\left(3 + \frac{10}{s+1}\right)}{8s^2 + s + 4}$$
$$v_o(s) = \frac{12 + \frac{40}{s+1}}{8s^2 + s + 4}$$

$$\frac{12 + \frac{40}{s+1}}{8s^2 + s + 4} = \frac{12}{8s^2 + s + 4} + \frac{40}{(s+1)(8s^2 + s + 4)}$$

$ \frac{12}{8s^2 + s + 4}$, can be expressed as
$$\frac{12}{8s^2 + s + 4} = \frac{Ds + E}{\left(s + \frac{1}{16}\right)^2 + \frac{127}{256}}$$

$$\frac{40}{(s+1)(8s^2 + s + 4)} = \frac{A}{s+1} + \frac{Bs + C}{\left(s + \frac{1}{16}\right)^2 + \frac{127}{256}}$$

% Step 5: Find the coefficients
Finding the denominator factors,
$$8s^2 + s + 4 = 8\left(s^2 + \frac{s}{8} + \frac{1}{2}\right)$$

$$s^2 + \frac{s}{8} + \frac{1}{2} = \left(s + \frac{1}{16}\right)^2 + \frac{1}{2} - \frac{1}{256} = \left(s + \frac{1}{16}\right)^2 + \frac{127}{256}$$

$$8s^2 + s + 4 = 8\left(\left(s + \frac{1}{16}\right)^2 + \frac{127}{256}\right) = 8\left(s + \frac{1}{16}\right)^2 + \frac{127}{32}$$

Solving for $A$, $B$, and $C$,
$$40 = A\left(8\left(-1 + \frac{1}{16}\right)^2 + \frac{127}{32}\right) = A\left(8\cdot\frac{225}{256} + \frac{127}{32}\right)$$
$$40 = A\left(\frac{225}{32} + \frac{127}{32}\right) = A\left(\frac{352}{32}\right) = 11A$$
$$A = \frac{40}{11}$$

% $$\text{Through a lengthy process of comparing coefficients, we get:}$$
$$B = -\frac{40}{11}$$
$$C = \frac{864}{11\sqrt{127}}$$

% Step 6: Write the partial fraction expansion
% $$\text{Write the partial fraction expansion:}$$
$$v_o(s) = \frac{40}{11}\frac{1}{s+1} + \frac{-40}{11}\frac{s}{\left(s + \frac{1}{16}\right)^2 + \frac{127}{256}} + \frac{864}{11\sqrt{127}}\frac{1}{\sqrt{\left(s + \frac{1}{16}\right)^2 + \frac{127}{256}}}$$

% Step 7: Find the inverse Laplace transform
Find the inverse Laplace transform of each term:
$$\frac{40}{11}\frac{1}{s+1} \rightarrow \frac{40}{11}e^{-t}$$

$$\frac{-40}{11}\frac{s}{\left(s + \frac{1}{16}\right)^2 + \frac{127}{256}}$$
$$\omega = \frac{\sqrt{127}}{16}, \text{ then:}$$
$$\frac{-40}{11}\frac{s}{\left(s + \frac{1}{16}\right)^2 + \omega^2} \rightarrow \frac{-40}{11}e^{-\frac{t}{16}}\cos(\omega t)$$

$$\frac{864}{11\sqrt{127}}\frac{1}{\sqrt{\left(s + \frac{1}{16}\right)^2 + \frac{127}{256}}}$$
$$\omega = \frac{\sqrt{127}}{16}$$
$$\frac{864}{11\sqrt{127}}\frac{1}{\sqrt{\left(s + \frac{1}{16}\right)^2 + \omega^2}} \rightarrow \frac{864}{11\sqrt{127}}e^{-\frac{t}{16}}\sin(\omega t)$$

% Step 8: Combine all terms to get v_o(t)
Combining all terms,
$$v_o(t) = \frac{40}{11}e^{-t} - \frac{40}{11}e^{-\frac{t}{16}}\cos\left(\frac{\sqrt{127}t}{16}\right) + \frac{864}{11\sqrt{127}}e^{-\frac{t}{16}}\sin\left(\frac{\sqrt{127}t}{16}\right)$$

\subsection*{16.44}
% \includegraphics[scale=0.9]{/Users/bryansebaraj/Desktop/Screenshot 2025-03-29 at 4.25.41 PM.png}

First note the initial conditions. For $t<0$, the inductor functions as a short circuit, yielding $v_c(0)=0$ and $i(0)=3$ A.

\noindent Also see that this circuit can be simplified and transformed into: 

\begin{center}
\includegraphics[scale=0.12]{/Users/bryansebaraj/Downloads/IMG_20410D708226-1.jpeg}
\end{center}

\noindent Solving for $V$ and $I$ in the above circuit yields the following:

$$-\frac9s + \frac{V}{\frac{100}{s}}+\frac{V}{8} + \frac{V}{4s} + \frac3s = 0$$

    $$I=\frac{V}{4s} + \frac3s$$


$$\frac6s = \left[\frac{s}{100} + \frac18 \frac{1}{4s}\right] \text{ V} = \frac{s^2 + 12.5s + 25}{100 s} \text{ V}$$

$$V(s)=\frac{600}{(s+2.5)(s+10)}$$

\noindent Therefore, $$I(s)=\frac{150}{s(s+2.5)(s+10)} + \frac3s=\frac{150 + 3(s+2.5)(s+10)}{s(s+2.5)(s+10)}$$

\noindent Performing partial fraction expansion and reconverting back into the time domain yields:

$$I(s)=\frac{A}{s} + \frac{B}{s+2.5} + \frac{C}{s+10}$$

$$A=(s)I(s)|_{s=0} = \frac{150 + 75}{25} = 9$$
$$B=(s+2.5)I(s)|_{s=-2.5} = \frac{150 + 0}{-2.5 * 7.5} = -8$$
$$C = (s+10)I(s)|_{s=-10} = \frac{150 + 0}{-10 * -7.5} = 2$$
\noindent Therefore, $$I(s)=\frac9s - \frac{8}{s+2.5} + \frac{2}{s+10}$$
\noindent Thus, $$i(t) = \mathbb{L}\{I(s)\}= u(t) \left( 9 - 8e^{-2.5t} + 2e^{-10t}\right) \text{ A}$$


\section*{Homework for April 1, 2025}

\subsection*{16.96}
% \includegraphics[scale=0.9]{/Users/bryansebaraj/Desktop/Screenshot 2025-03-29 at 4.26.02 PM.png}

Given that $V_o$ is the voltage across $R$, KCL yields
$$I_s=\frac{V_o}{R} + sCV_o + \frac{V_o}{sL}=V_o \left(\frac{1}{R} + sC + \frac{1}{sL}\right)$$
Solving for $V_o$,
$$V_o=\frac{I_s}{\frac{1}{R} + sC + \frac{1}{sL}}$$
$$V_o=\frac{sLI_sR}{sL + R + s^2LRC}$$

Given that $I_o = \frac{V_o}{R}$, 
$$I_o=\frac{sLI_s}{sL + R + s^2LRC}$$
$$H(s) = \frac{I_o}{I_s}=\frac{sL}{sL + R + s^2LRC}=\frac{\frac{s}{RC}}{s^2 + \frac{s}{RC} + \frac{1}{LC}}$$

This yields the following roots:

$$s_{1,2} = \frac{-\frac{1}{RC} \pm \sqrt{\left(\frac{1}{RC}\right)^2 - 4\left(\frac{1}{LC}\right)}}{2}=-\frac{1}{2RC} \pm \sqrt{\left(\frac{1}{2RC}\right)^2 - \left(\frac{1}{LC}\right)}$$

Given that R, L, and C are positive, the roots $s_1$ and $s_2$ must be negative.
Therefore, the system is stable.

\end{document}

