\documentclass{article}
\usepackage{graphicx}
\usepackage{amsfonts}
% \usepackage{fancyhdr}
\usepackage[a4paper, margin=0.5in]{geometry}
\usepackage{amsmath}
\usepackage{amssymb}
\usepackage{enumitem}


% \usepackage{amsmath}

\title{Week 4 Homework}
\author{Bryan SebaRaj \\[0.7em] Professor Hong Tang \\[0.7em]  EENG 203 - Circuits and System Design}
\date{February 11, 2025}
\begin{document}



\maketitle

\section*{Homework for February 4, 2025}

\subsection*{14.35}
% \includegraphics[scale=0.4]{/Users/bryansebaraj/Desktop/Screenshot 2025-02-08 at 7.18.47 PM.png}

In a parallel RLC circuit,

\begin{enumerate}[label=(\alph*)]
    \item $\omega_o=\frac{1}{\sqrt{LC}}=\frac{1}{8 mH \cdot 60 \mu F}=1443.38$ rad/s
    \item $B=\frac{1}{RC}=\frac{1}{5 k\Omega \cdot 60 \mu F}=3.33$ rad/s
    \item $Q=\omega_oRC=1443.38 \text{ rads/s } \cdot 5 k\Omega \cdot 60 \mu F =  433.01$
\end{enumerate}

\subsection*{14.40}
% \includegraphics[scale=0.4]{/Users/bryansebaraj/Desktop/Screenshot 2025-02-08 at 7.19.04 PM.png}

In a parallel resonance circuit,

\begin{enumerate}[label=(\alph*)]
    \item From $B=\frac{1}{RC}$, $$C=\frac{1}{BR}=\frac{1}{(\omega_2-\omega_1)R}=\frac{1}{2\pi(f_2-f_1)R}=\frac{1}{2\pi(4 kHz) \cdot 2 k\Omega}=19.894 \text{ nF}$$
\item From $\omega_0=\frac{1}{\sqrt{LC}}$, $$L=\frac{1}{\omega^2_oC}=\frac{1}{\left( \frac{\omega_1+\omega_2}{2}\right)^2 \cdot C}=\frac{1}{\left( 2\pi f_0 \cdot \right)^2 \cdot C}=164.42 \mu H$$
\item $\omega_0=\frac{\omega_1+\omega_2}{2}=2\pi\frac{f_1+f_2}{2}=2\pi \cdot 88 kHz=552.92$ krad/s
\item $B=\omega_2-\omega_1= 2\pi (f_2-f_1)=8\pi$ krad/s = 25.13 krad/s
\item $Q=\frac{\omega_0}{B}=\frac{176\pi}{8\pi}=22$
\end{enumerate}



\section*{Homework for February 6, 2025}

\subsection*{14.50}
% \includegraphics[scale=0.9]{/Users/bryansebaraj/Desktop/Screenshot 2025-02-08 at 7.19.45 PM.png}

Note that $H(\omega)=\frac{V_o}{V_i}=\frac{j\omega L}{R + j\omega L}$

$H(0)=\frac{0}{R + 0}=0$ and $H(\infty)=\frac{j\infty L}{R + j\infty L}=1$. Therefore, the given circuit is a high pass filter.

Calculating the corner frequency, using $H(\omega_c)=\frac{1}{\sqrt{2}}$,
$$H(\omega_c)=\frac{1}{\sqrt{1+\left(\frac{R}{\omega_cL}\right)^2}}$$
$$\sqrt{1+\left(\frac{R}{\omega_c L}\right)^2} = \sqrt{2}$$
$$\frac{R}{\omega_cL} = \sqrt{1}=1$$
$$\omega_c=\frac{R}{L}$$
$$2\pi f_c=\frac{R}{L}$$
$$f_c=\frac{R}{2\pi L}=318.3 \text{ Hz} $$

\subsection*{14.57}
% \includegraphics[scale=0.9]{/Users/bryansebaraj/Desktop/Screenshot 2025-02-08 at 7.20.13 PM.png}

\begin{enumerate}[label=(\alph*)]
    \item   \includegraphics[scale=0.1]{/Users/bryansebaraj/Downloads/IMG_BEF698674C59-1.jpeg} \\ 
        First, the input impedance can be calculated as $$Z(s)=R + \frac{\frac{1}{sC}\left(R + \frac{1}{sC} \right)}{\frac{1}{sC} + R + \frac{1}{sC}}$$
        $$Z(s) = R+ \frac{RsC + 1}{sC \cdot sC \cdot \left( R + \frac{2}{SC}\right)}=R+\frac{RsC + 1}{2sC + s^2RC^2}$$
        Recall that $$I=\frac{V_s}{Z}$$
        $$I_1=\frac{\frac{1}{sC}}{\frac{1}{sC} + R + \frac{1}{sC}}I = \frac{V_s}{Z(sRC + 2)}=\frac{V_s}{\frac{1 + 3RsC + s^2R^2C^2}{sC}}=\frac{V_ssC}{1 + 3RsC + s^2R^2C^2}$$
        $$V_o=I_1R=\frac{RV_ssC}{1 + 3RsC + s^2R^2C^2}$$
        $$\frac{V_o}{V_s}=H(s)=\frac{RsC}{1 + 3RsC + s^2R^2C^2}$$
        Dividing the numerator and denominator by $R^2C^2$,
        $$H(s)=\frac{\frac{s}{RC}}{\frac{1}{R^2C^2} + \frac{3s}{RC} + s^2}$$
        Rewriting to balance the $\frac{s}{RC}$ coefficient, 
        $$H(s)=\frac{1}{3}\left( \frac{\frac{3s}{RC}}{\frac{1}{R^2C^2} + \frac{3s}{RC} + s^2} \right)$$
        Therefore, $\omega^2_0=\frac{1}{R^2C^2}$, so $$\omega_0=\frac{1}{RC}=1 \text{ rad/s}$$
        $$B_{\text{bandpass filter}}=\frac{3}{RC}= 3 \text{ rad/s}$$


    \item \includegraphics[scale=0.1]{/Users/bryansebaraj/Downloads/IMG_C2687C82924F-1.jpeg}
        $$Z(s)=sL + \frac{R(R + sL)}{R + sL + R}=\frac{R^2 + 3sRL + s^2L^2}{2R+sL}$$
        Recall that $$I=\frac{V_s}{Z}$$
        $$I_1=\frac{R}{sL + 2R}I= \frac{V_sR}{Z(sL + 2R)}=\frac{V_sR}{R^2+3sRL+s^2L^2}$$
        $$V_o=I_1sL=\frac{V_sRL}{R^2+3sRL+s^2L^2}$$
        $$\frac{V_o}{V_s}=H(s)=\frac{R}{R^2+3sRL+s^2L^2}$$
        Dividing the numerator and denominator by $L^2$,
        $$H(s)=\frac{\frac{sR}{L}}{\frac{R^2}{L^2} + \frac{3sR}{L} + s^2}$$
        Rewriting to balance the $\frac{sR}{L}$ coefficient,
        $$H(s)=\frac{1}{3}\left( \frac{\frac{3sR}{L}}{\frac{R^2}{L^2} + \frac{3sR}{L} + s^2} \right)$$
        Therefore, $\omega^2_0=\frac{R^2}{L^2}$, so $$\omega_0=\frac{R}{L}=1 \text{ rad/s}$$
        $$B_{\text{bandpass filter}}=\frac{3R}{L}= 3 \text{ rad/s}$$
\end{enumerate}




\end{document}

