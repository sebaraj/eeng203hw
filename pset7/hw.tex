\documentclass{article}
\usepackage{graphicx}
\usepackage{amsfonts}
% \usepackage{fancyhdr}
\usepackage[a4paper, margin=0.5in]{geometry}
\usepackage{amsmath}
\usepackage{amssymb}
\usepackage{enumitem}


% \usepackage{amsmath}

\title{Week 7 Homework}
\author{Bryan SebaRaj \\[0.7em] Professor Hong Tang \\[0.7em]  EENG 203 - Circuits and System Design}
\date{March 4, 2025}
\begin{document}



\maketitle

\section*{Homework for February 25, 2025}

\subsection*{16.13}
\includegraphics[scale=0.9]{/Users/bryansebaraj/Desktop/Screenshot 2025-02-28 at 4.32.40 PM.png}

Given the circuit show in Figure 16.36, suppose that $v_s=4u(t)$, $L=1$ H, the capacitor $C$ is equivalent to $C= \frac18$ H, $R_1=2$ $\Omega$, and $R_2=4$ $\Omega$. Find $v_x$

We first calculate

$$\frac{V_x - \frac4s}{s}+\frac{V_x}{2}+\frac{V_x}{4+\frac8s}=0$$

Multiplying by $s(4s+8)$, we get

$$V_x(4s+8) - \frac{16s+32}{s}+V_x(s(2s+4)) + V_xs^2$$

$$V_x(3s^2 + 8s + 8) = \frac{16s+32}{s}$$

$$V_x = \frac{16s+32}{s(3s^2 + 8s + 8)}$$

Separating into partial fractions, we get

$$V_x = 16\left(\frac{A}{s} + \frac{B}{s + \left( \frac{4 + j\sqrt{8}}{3} \right)} + \frac{C}{s + \left( \frac{4 - j\sqrt{8}}{3} \right)}\right)$$

Solving for $A$, $B$, and $C$, we get

$$A = \frac{1}{4}$$

$$B = -\frac18$$

$$C = -\frac18$$

Therefore,

$$V_x = \frac{4}{s} - \frac{2}{s + \left( \frac{4 + j\sqrt{8}}{3} \right)} - \frac{2}{s + \left( \frac{4 - j\sqrt{8}}{3} \right)}$$

Taking the inverse Laplace transform, we get

$$v_x = 4u(t) - 4e^{-\frac43t}\cos\left(\frac{2\sqrt{2}}{3}t\right)u(t) \ \text{V}$$




\subsection*{16.15}
\includegraphics[scale=0.9]{/Users/bryansebaraj/Desktop/Screenshot 2025-02-28 at 4.32.58 PM.png}

\includegraphics[scale=0.9]{/Users/bryansebaraj/Desktop/Screenshot 2025-02-28 at 4.33.07 PM.png}

Let $R_o = R || 60$ $\Omega$. Converting the circuit to the $s-domain$, 

$$T(s) = R_o + \frac{1}{0.01s} + 4s = R_o + \frac{100}{s} + 4s= \frac{4s^2 + sR_o + 100}{s}$$

Therefore, 
$$s_\pm = \frac{-R_o \pm \sqrt{R_o - 1600}}{2}$$

the system is critically damped when $R_o = 40$. 

$$R_0 = \frac{R \times 60}{R + 60}$$

$$20R = 2400$$

$$R = 120 \ \Omega$$

\end{document}
