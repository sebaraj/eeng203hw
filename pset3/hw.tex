\documentclass{article}
\usepackage{graphicx}
\usepackage{amsfonts}
% \usepackage{fancyhdr}
\usepackage[a4paper, margin=1in]{geometry}
\usepackage{amsmath}
\usepackage{amssymb}
\usepackage{enumitem}


% \usepackage{amsmath}

\title{Week 3 Homework}
\author{Bryan SebaRaj \\[0.7em] Professor Hong Tang \\[0.7em]  EENG 203 - Circuits and System Design}
\date{February 4, 2025}
\begin{document}



\maketitle

\section*{Homework for January 28, 2025}

\subsection*{14.3}
\includegraphics[scale=0.4]{/Users/bryansebaraj/Desktop/Screenshot 2025-02-03 at 7.02.05 PM.png}

Note that, $$C_1  = 0.1 F \rightarrow \frac{1}{j\omega C}= \frac{1}{s * 0.1F}  = \frac{10}{s}$$
$$C_2  = 0.2 F \rightarrow \frac{1}{j\omega C}= \frac{1}{s * 0.2 F} = \frac{5}{s}$$

We can then use the RC circuit below to solve for the transfer function:

\includegraphics[scale=0.1]{/Users/bryansebaraj/Downloads/IMG_4FC8BFDACB27-1.jpeg}

$$Z_{in}(\omega)= \frac{Z_1 Z_2}{Z_1 + Z_2} \text{ where } Z_1 = \frac{10}{s} \text{ and } Z_2 = \frac{5}{s} + 5$$

So, $$Z_{in}(\omega) = \frac{\frac{10}{s} (\frac{5}{s} + 5)}{\frac{15}{s} + 5} = \frac{50 + 50s}{15s + 5s^2}=\frac{10 + 10s}{3s + s^2}$$

At $V_1$, $$V_1 = \frac{Z_{in}(\omega)}{Z_{in}(\omega) + 2 \Omega}V_i$$
$$V_o = \frac{5 \Omega}{5 \Omega + \frac{5}{s}}V_1 = \frac{5 \Omega}{5 \Omega + \frac{5}{s}}\frac{Z_{in}(\omega)}{Z_{in}(\omega) + 2 \Omega}V_i$$
$$\frac{V_o}{V_i}=\frac{5 \Omega}{5 \Omega + \frac{5}{s}}\frac{Z_{in}(\omega)}{Z_{in}(\omega) + 2 \Omega}$$

Solving for $H(s)$,

$$H(s) = \frac{V_o(s)}{V_i(s)} = \frac{5}{5 + \frac{5}{s}}\frac{\frac{10 + 10s}{3s + s^2}}{\frac{10 + 10s}{3s + s^2} + 2 } = \frac{s}{1+s}\frac{10 (1 + s)}{10 + 10 s + 6s + 2s^2}=\frac{10s}{2s^2 + 16s + 10}=\frac{5s}{s^2 + 8s + 5}$$
$$H(s) = \frac{5s}{s^2 + 8s + 5}$$

\subsection*{14.7}
\includegraphics[scale=0.4]{/Users/bryansebaraj/Desktop/Screenshot 2025-02-03 at 7.02.21 PM.png}

Using the formula, $$20 \text{log}_{10}|H(\omega)| = H_{dB} \rightarrow |H(\omega)| = 10^{H_{db} / 20}$$

(a) $$|H(\omega)| = 10^{H_{db} / 20} = 10^{0.0025}=1.00577$$

(b) $$|H(\omega)| = 10^{H_{db} / 20} = 10^{-0.31}=0.48978$$

(c) $$|H(\omega)| = 10^{H_{db} / 20} = 10^{5.235}=1.71791 \cdot 10^5$$


\section*{Homework for January 30, 2025}

\subsection*{14.9}
\includegraphics[scale=0.4]{/Users/bryansebaraj/Desktop/Screenshot 2025-02-03 at 7.02.31 PM.png}

$$H(\omega)=\frac{10}{10(1 + j\omega)(1 + \frac{j\omega}{10})}=\frac{1}{(1 + j\omega)(1 + \frac{j\omega}{10})}$$

Therefore, there are two poles at 1 and 10, and no zeros.

% Splitting $H(\omega)$ into $H_{dB}$ and $\phi$,

% $$H_{dB} = 20 log_{10}(1) - 20log_{10}(1 + j\omega)-20log_{10}(1 + \frac{j\omega}{10})$$

\includegraphics[scale=0.15]{/Users/bryansebaraj/Downloads/IMG_C9BF83A9485B-1.jpeg}

\subsection*{14.16}
\includegraphics[scale=0.5]{/Users/bryansebaraj/Desktop/Screenshot 2025-02-03 at 7.02.44 PM.png}

$$H(s)=\frac{1.6}{16s(\frac{s^2}{16}+\frac{s}{16} + 1)}=\frac{0.1}{s(1+\frac{s^2}{16}+\frac{s}{16})}$$

\includegraphics[scale=0.20]{/Users/bryansebaraj/Downloads/IMG_6B6DE9D7BA73-1.jpeg}



\end{document}

