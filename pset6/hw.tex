\documentclass{article}
\usepackage{graphicx}
\usepackage{amsfonts}
% \usepackage{fancyhdr}
\usepackage[a4paper, margin=0.5in]{geometry}
\usepackage{amsmath}
\usepackage{amssymb}
\usepackage{enumitem}


% \usepackage{amsmath}

\title{Week 6 Homework}
\author{Bryan SebaRaj \\[0.7em] Professor Hong Tang \\[0.7em]  EENG 203 - Circuits and System Design}
\date{February 25, 2025}
\begin{document}



\maketitle

\section*{Homework for February 18, 2025}

\subsection*{15.5}
% \includegraphics[scale=0.5]{}

\subsection*{15.8}
% \includegraphics[scale=0.5]{}

\subsection*{15.9}
% \includegraphics[scale=0.5]{}

\section*{Homework for February 20, 2025}

\subsection*{15.21}
% \includegraphics[scale=0.9]{/Users/bryansebaraj/Desktop/Screenshot 2025-02-22 at 10.36.00 PM.png}

Let the period be $T=2\pi$.
From inspection, $$f_1(t) = (1 - \frac{t}{2\pi})(u(t)-i(t-2\pi))=u(t) - \frac{t}{2\pi}u(t) + \frac{t - 2\pi}{2\pi}u(t- 2\pi)$$

Using the common functions of a Laplace transform,

$$\mathcal{L}\{f_1(t)\} = F_1(s) = \frac{1}{s} - \frac{1}{2\pi s^2} + \frac{e^{-2\pi s}}{2\pi s^2} = \frac{2\pi s - 1 + e^{-2\pi s}}{2\pi s^2}$$

Projecting the Laplace transform from a single period to the entire function, 

$$F(s) = \frac{F_1(s)}{1 - e^{-2\pi s}}=\frac{2\pi s - 1 + e^{-2\pi s}}{2\pi s^2 - 2\pi s^2 e^{-2\pi s}}$$

\subsection*{15.22}
% \includegraphics[scale=0.9]{/Users/bryansebaraj/Desktop/Screenshot 2025-02-22 at 10.36.07 PM.png}

\begin{enumerate}[label=(\alph*)]
    \item Let the period be $T=1$.
        From inspection, $$g_1(t)  = 2t(u(t)-u(t-1)) = 2tu(t)-2(tu(t-1))=2tu(t)+2u(t-1)-2(t-1)u(t-1)$$
        Using the common functions of a Laplace transform,
        $$\mathcal{L}\{g_1(t)\} = G_1(s) = \frac{2}{s^2} + \frac{2e^{-s}}{s} - \frac{2e^{-s}}{s^2} = \frac{2 + 2se^{-s} - 2e^{-s}}{s^2}$$
        Projecting the Laplace transform from a single period to the entire function,
        $$G(s) = \frac{G_1(s)}{1 - e^{-s}} = \frac{2 + 2se^{-s} - 2e^{-s}}{s^2 - s^2 e^{-s}}$$
    \item Let the period be $T=2$. Define the periodic triangular wave as $h_0$, where $h=h_0 + u(t)$.
        From inspection, $$h_1(t) = 2tu(t)+2(1-2t)u(t-1)+2u(t-1) +2(t-2)u(t-2)= 2tu(t) + 4u(t-1) - 4tu(t-1)+2(t-2)u(t-2)$$
        $$=2tu(t)  - 4(t-1)u(t-1)+2(t-2)u(t-2)$$
        Using the common functions of a Laplace transform,
        $$\mathcal{L}\{h_1(t)\} = H_1(s) = \frac{2}{s^2} - \frac{4e^{-s}}{s^2} + \frac{2e^{-2s}}{s^2}= \frac{2  - 4e^{-s}+2e^{-2s}}{s^2}=\frac{2(1-e^{-s})^2}{s^2}$$
        Finding $H_0$,
        $$H_0(s) = \frac{2(1-e^{-s})^2}{s^2(1-e^{-2s})}$$
        Projecting the Laplace transform from a single period to the entire function,
        $$H(s) = \frac{2(1-e^{-s})^2}{s^2(1-e^{-2s})} + \frac{1}{s}$$

\end{enumerate}


\subsection*{15.27}
% \includegraphics[scale=0.9]{/Users/bryansebaraj/Desktop/Screenshot 2025-02-22 at 10.36.18 PM.png}

\begin{enumerate}[label=(\alph*)]
    \item Trivially using the reverse transforms, 
        $$f(t) = \mathcal{L}^{-1}\{F(s)\} = u(t) + 2e^{-t}u(t)$$
    \item See that $$G(s)=\frac{3(s+4)-11}{s+4}= 3 - \frac{11}{s+4}$$
        $$g(t) = \mathcal{L}^{-1}\{G(s)\} = 3\delta(t) - 11e^{-4t}u(t)$$
    \item See that $$H(s)=\frac{A}{s+1}+\frac{B}{s+3}, \quad \exists A,B \in \mathbb{Z}$$
        The poles of $H(s)$ can be calculated as 
        $$A=(s+1)H(s)\big|_{s=-1}=2$$
        $$B=(s+3)H(s)\big|_{s=-3}=-2$$
        Thus, $$H(s)=\frac{2}{s+1}-\frac{2}{s+3}$$
        $$h(t) = \mathcal{L}^{-1}\{H(s)\} = 2e^{-t}u(t) - 2e^{-3t}u(t)$$
    \item See that $$J(s)=\frac{A}{s+2}+\frac{B}{(s+2)^2}+\frac{C}{s+4}, \quad \exists A,B,C \in \mathbb{Z}$$
        The poles of $J(s)$ can be calculated as 
        $$B=(s+2)^2J(s)\big|_{s=-2}=6$$
        $$C=(s-4)J(s)\big|_{s=4}=3$$
        Note that $A$ canot be determined in this manner, so we can multiply both sides of $J(s)$ by $(s+2)^2(s+4)$ and rewrite it as
        $$12=A(s+2)(s+4)+B(s+4)+C(s+2)^2$$
        Substituting in $B$ and $C$,
        $$12=A(s+2)(s+4)+6(s+4)+3(s+2)^2$$
        Without expanding, we can see that $A=-3$. Thus,
        $$J(s)=\frac{-3}{s+2}+\frac{6}{(s+2)^2}+\frac{3}{s+4}$$
        $$j(t) = \mathcal{L}^{-1}\{J(s)\} = -3e^{-2t}u(t) + 6te^{-2t}u(t) + 3e^{-4t}u(t)$$

\end{enumerate}



\subsection*{15.30}
% \includegraphics[scale=0.9]{/Users/bryansebaraj/Desktop/Screenshot 2025-02-22 at 10.36.23 PM.png}

\begin{enumerate}[label=(\alph*)]
    \item See that we can split up $F_1(s)$ such that $$F_1(s)=\frac{A}{s} + \frac{Bs+C}{s^2+2s+5}$$
        Multiplying both sides by $s(s^2+2s+5)$, we get
        $$6s^2+8s+3 = A(s^2+2s+5)+Bs^2+Cs$$
        Setting the coefficients of both sides equal to each other, see that
        $$\text{for } s^2, 6 = A + B$$
        $$\text{for } s^1, 8 = 2A + C$$
        $$\text{for } s^0, 3 = 5A$$
        Thus, $A=\frac{3}{5}$, $B=\frac{27}{5}$, and $C=\frac{34}{5}$.
        Substituting back into $F_1(s)$,
        $$F_1(s)=\frac{3}{5s} + \frac{27s+34}{5(s^2+2s+5)}=\frac{3}{5s}+\frac{27(s+1)+7}{5[(s+1)^2 + 4]}$$
        $$f_1(t)=\frac{3}{5}+\frac{27}{5}e^{-t}cos(2t)+\frac{7}{5}s^{-t}sin(2t)$$
    \item Splitting up $F_2(s)$, $$F_2(s)=\frac{A}{s+1}+\frac{B}{(s+1)^2}+\frac{C}{s+4}$$
        Multiplying both sides by $(s+1)^2 (s+4)$, we get
        $$6s^2+8s+3 = A(s+1)(s+4)+B(s+4)+C(s+1)^2$$
        Setting the coefficients of both sides equal to each other, see that
        $$\text{for } s^2, 1 = A+C$$
        $$\text{for } s^1, 5 = 5A+B+2C$$
        $$\text{for } s^0, 6 = 4A+4B+C$$
        Solving the system of equations, 
        $$14=16A+7C$$
        $$7=9A$$
        $$A=\frac{7}{9}, C=\frac{2}{9}, B = \frac{2}{3}$$
        Substituting back into $F_2(s)$,
        $$F_2(s)=\frac{7}{9(s+1)}+\frac{2}{3(s+1)^2}+\frac{2}{9(s+4)}$$
        $$f_2(t)=\frac{7}{9}e^{-t}u(t)+\frac{2}{3}te^{-t}u(t)+\frac{2}{9}e^{-4t}u(t)$$
    \item Splitting up $F_3(s)$, $$F_3(s)=\frac{A}{s+1}+\frac{Bs+C}{s^2+4s+8}$$
        Multiplying both sides by $(s+1)(s^2+4s+8)$, we get
        $$10 = A(s^2+4s+8)+Bs^2+Bs+Cs+C$$
        Setting the coefficients of both sides equal to each other, see that
        $$\text{for } s^2, 0 = A+B$$
        $$\text{for } s^1, 0 = 4A+B+C$$
        $$\text{for } s^0, 10 = 8A+C$$
        Solving the system of equations,
        $$3A+C=0$$
        $$5A=10$$
        $$A=2, B=-2, C=-6$$
        Substituting back into $F_3(s)$,
        $$F_3(s)=\frac{2}{s+1}-\frac{2s+6}{s^2+4s+8}$$
        $$=\frac{2}{s+1}-\frac{2(s+1)}{(s+1)^2+4}-\frac{4}{(s+1)^2 + 4}$$
        $$f_3(t)=2e^{-t}u(t)-2e^{-t}cos(2t)u(t) - 2e^{-t}sin(2t)u(t)$$
\end{enumerate}




\end{document}

