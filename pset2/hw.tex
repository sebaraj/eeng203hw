\documentclass{article}
\usepackage{graphicx}
\usepackage{amsfonts}
% \usepackage{fancyhdr}
\usepackage[a4paper, margin=1in]{geometry}
\usepackage{amsmath}
\usepackage{amssymb}
\usepackage{enumitem}


% \usepackage{amsmath}

\title{Week 2 Homework}
\author{Bryan SebaRaj \\[0.7em] Professor Hong Tang \\[0.7em]  EENG 203 - Circuits and System Design}
\date{January 28, 2025}
\begin{document}



\maketitle

\section*{Homework for January 21, 2025}

\subsection*{8.14}

\includegraphics[scale=0.3]{/Users/bryansebaraj/Downloads/Prob 8.14.jpg}

When moving the switch from position A to position B at $t=0$, $v(0)=v(0^-)=0$ since voltage across a capacitor cannot change instantaneously, and $i_L(0)=\frac{80 V}{30 \Omega + 10 \Omega}=2 A$. \\

After $t=0$, the circuit is a source-free series RCL circuit. 

$$\omega_0 = \frac{1}{\sqrt{LC}}=\frac{1}{\sqrt{4 H \cdot 0.25 F}}=1 \text{ Hz}$$

$$\alpha = \frac{R}{2L}=\frac{10 \Omega}{1 \cdot 4 H}=1.25 \text{ Hz}$$

Since $\alpha > \omega_0$, the circuit is overdamped.

Therefore, $$s_{\pm}= - \alpha \pm \sqrt{\alpha^2 - \omega^2_0}=-1.25 \pm \sqrt{1.25^2 - 1^2}=-1.25 \pm 0.75$$

$$s_{+} = -0.5 \text{ and } s_{-} = -2$$

Thus, $$v(t)=A_1^{-2t} + A_2e^{-0.5t}$$

At $t=0$, $v(0)=A_1+A_2=0$ and $i_C(0)=C\frac{dv(0)}{dt}=-2$

Rewriting in terms of $\frac{dv}{dt}$, $$\frac{dv(0)}{dt}=\frac{-2}{C}=\frac{-2}{0.25 F} = -8 \text{ A}$$

Differentiating $v(t)$,

$$\frac{dv(t)}{dt}=-2A_1e^{-2t} - 0.5A_2e^{-0.5t}$$

At $t=0$, $\frac{dv(0)}{dt}=-2A-0.5B=-8$

Substituting $A_1 = 4 - 0.25 A_2$ into $v(0)= A_1+A_2 = 0$, 

$$4 - 0.25A_2 + A_2 = 0$$

$$A_2 = -5.3333$$

$$A_1 = 5.3333$$

Thus, $$v(t) = 5.3333 e^{-2t} - 5.3333 e^{-0.5t} \text{ V, }, \forall t \geq 0$$

\subsection*{8.31}

\includegraphics[angle=180,scale=0.3]{/Users/bryansebaraj/Downloads/Prob 8.31.jpg}

At $t=0^-$, the $2u(t)$ current source acts as an open circuit, and the $1 $ F capcitor acts as an open circuit (via trivial properties of a step function and capacitor with a source, respectively). The inductor at steady state ($t=0^-$) behaves as a short circuit. \\ 

The circuit at $t=0^-$:

\includegraphics[scale=0.1]{/Users/bryansebaraj/Downloads/IMG_4C83FA4F8735-1.jpeg}

The circuit at $t=0^+$: 

\includegraphics[scale=0.1]{/Users/bryansebaraj/Downloads/IMG_C9E306F63349-1.jpeg}

The current through the entire circuit at $t=0^-$, $$i(0^-)= i(0^+)=\frac{V_{source}(0^-)}{R_{total}}=\frac{50 V}{10 \Omega + 40 \Omega}=1 A$$

Calculating the voltage at the node with the open circuit capacitor (i.e. the 40 $\Omega$ resistor), 

$$v(0^-) = v(0^+)_C= i \cdot R_1 = 1 A \cdot 40 \Omega= 40 \text{ V}$$

By KCL, the current at the node with the current source, 40 $\Omega$ resistor, and 0.5 H inductor is 

$$i(0^+) + i_1= 1 + i_1 = 2 \text{, so } i_1 = 1$$

Through KVL, the volage across the inner loop is 

$$-v_L(0^+) + 40i_1 + v_C(0^+)=0$$

$$v_L(0^+)= 40 \cdot 1 + 40 \text{ V} = 80 \text{ V}$$

Thus, $v_L(0^+)=80 \text{ V and } v_C(0^+)=40 \text{ V}$

\section*{Homework for January 23, 2025}

\subsection*{8.23}

\includegraphics[scale=0.3]{/Users/bryansebaraj/Downloads/Problem 8.23.jpg}

This is a source-free, parallel RCL circuit. Let the total capacitance be $C_{total}=C + 10 \text{ mF }$.

In a parallel circuit, $\alpha = \frac{1}{2RC_{total}}$ and $\omega_0 = \frac{1}{\sqrt{LC_{total}}}$.

Given that $\alpha = 1$, $$\frac{1}{2RC_{total}}=1$$

$$C_{total}=\frac{1}{2 \cdot 10 \Omega}= 50 \text{ mF}$$

Thus, $C=C_{total} - 10 \text{ mF} = 40 \text{ mF}$.


\subsection*{8.47}

\includegraphics[scale=0.3]{/Users/bryansebaraj/Downloads/Problem 8.47.jpg}

At $t=0^-$, the circuit is a RLC circuit with a current source, with 

$$i_L(0^-)=i_{source}\frac{r_1}{r_1 + r_2}$$ where $r_1$ and $r_2$ are the $5 \Omega$ and $10 \Omega$ resistors, respectively.

$$i_L(0^-)=3 \text{ V } \frac{5 \Omega }{5 \Omega * 10 \Omega } = 1 \text{ A}$$

Since the 10 mF capacitor will function as an open circuit at $t = 0^-$, $$v_o(0)=0$$

At $t>0$, the 10 $\Omega$ resistor will function as a short circuit, creating a parallel RLC circuit. 

$$\alpha = \frac{1}{2RC}=\frac{1}{2\cdot 5 \Omega \cdot 10 \text{ mF}}=10 \text{ Hz}$$

$$\omega_0 = \frac{1}{\sqrt{LC}}=\frac{1}{\sqrt{1 H \cdot 10 { mF}}}=10 \text{ Hz}$$

Note that $\alpha = \omega_0$, so the circuit is critically damped.

$$s_{\pm} = -\alpha \pm \sqrt{\alpha^2 - \omega^2} = -10 \pm \sqrt{10^2 - 10^2} = 10$$

Therefore, $$i(t) = i_s + (A_1t + A_2)e^{-\alpha t} = 3 + (A_1t + A_2)e^{-10 t}$$

At $t=0$, $i(0)=3 + A_2 = 1$, so $$A_2 = -2$$

$$v_o = L \frac{di}{dt} = L [A_1e^{-10t} + (-10(A_1t+A_2)e^{-10t})] $$

$$v_o(0) = A_1 - 10A_2 = 0$$

$$A_1 = -20$$

Thus, $$v_o(t)=-20e^{-10t}+200te^{-10t}+20e^{-10t}$$

$$v_o(t)=200te^{-10t} \text{ V}$$

\end{document}

